\documentclass{article}

\usepackage[dvinames]{xcolor}
% Measurements are taken directly from the guide
\usepackage[top=1.7in,left=1in,bottom=1in,right=1in]{geometry}
\usepackage{graphicx}
\usepackage[colorlinks=true,
            pdfborder={0 0 0},
            ]{hyperref}
\usepackage{lipsum}
\usepackage[absolute]{textpos}
\usepackage{tikz}
\usetikzlibrary{calc}

% No paragraph indentation
\parindent0pt
\setlength{\parskip}{0.8\baselineskip}
\raggedright
\pagestyle{empty}
% Define SI official colors
\definecolor{UBC_blue}{RGB}{13,40,86}
\definecolor{SIgray}{HTML}{5e6a71}
% Ensure consistency in the footer
\urlstyle{rm}

\usepackage{fancyhdr}
\pagestyle{fancy}
\fancyfoot{}
\fancyhead{}
\fancyfoot[C]{ \thepage}

\setcounter{page}{1}
\renewcommand{\thepage}{Noonan, Cover Letter--- \arabic{page}}

\renewcommand{\footrulewidth}{0pt}
\renewcommand{\headrulewidth}{0pt}
\fancyfoot{}
\fancyfoot[L]{%
    {\footnotesize\color{SIgray}
Department of Biology, Irving K. Barber Faculty of Science, University of British Columbia, Kelowna, B.C., Canada \\[-0.1\baselineskip]
T: 1(250) 258-1118  $\mid$  \href{mailto:michael.noonan@ubc.ca}{michael.noonan@ubc.ca} $\mid$ \url{https://biology.ok.ubc.ca/about/contact/michael-j-noonan/}
}\color{black}}



%Cover Letter spends too much time going through individual papers you've written. I think it is fine to mention these accomplishments and their significance, but I think this part of the letter could/should be substantially condensed. I also think that somewhere in your app, possibly here in the cover letter, you need to identify more potential linkages to faculty in the dept. You want to give them the impression that your are a quant person that they could benefit from and work with. You might also considering adding a personal note about your desire to return to Canada, just so they don't miss that you're Canadian. 

\begin{document}

% -------------------------------------------------------
% Add logo, the text under the blue line, and the line itself
\begin{textblock*}{2in}[0.3066,0.39](1.6in,1.05in)
    \includegraphics[width=2in]{Lab_Logo_Trans.png}
\end{textblock*}
\begin{textblock*}{6.375in}(1.5in,1.1in)   % 6.375=8.5 - 1.5 - 0.625
    \hfill \color{SIgray} Dr.~Michael Noonan, Assistant Professor\\
    \hfill Department of Biology, UBC-Okanagan

\end{textblock*}
\begin{tikzpicture}[remember picture,overlay]
    \draw[color=UBC_blue,line width=1pt] (current page.north west)+(1.05in,-1.6in) -- ($(-0.625in,-1.6in)+(current page.north east)$);
\end{tikzpicture}


\color{black}
\vspace{20pt}
\hfill Kelowna, December 20, 2022

Dear Editors,

We are pleased to submit our manuscript entitled ``Rethinking the Road Effect Zone: The Probabilistic Effect of Roads on Ecosystems'' for consideration as a Note in American Naturalist.

Roads are important for human socio-economic growth, yet, while the area that roads occupy might be small, they carry substantial ecological impacts. The impacts of roads are so marked that they have spurred entire sub-disciplines of theoretical, ecological, and conservation research. Roads can cause a broad range of ecological impacts, but their effects are usually considered strongest with increasing proximity to the road surface. This relationship between the strength of effect and proximity to the road forms the basis for the concept of the  `Road Effect Zone', which is now a foundational theory in road ecology research. Ecologists and conservation practitioners regularly quantify road effect zones for different species, which are then used to make conservation recommendations. Although the concept of the road effect zone has proven useful in measuring patterns of biodiversity loss, it is a static measure that does account for the dynamic effects animal movement and species interactions. This is a noteworthy limitation. For instance, the effects of road mortality on highly vagile animals might decrease nutrient transfer, prey densities, and/or grazing pressure hundreds or even thousands of kilometers away from the road, yet these effects would not be captured within the current Road Effect Zone framework.

In this short note we re-frame the road effect as a joint probability of independent events and introduce a more mechanistic probabilistic road effect that describes the broader ecological impacts that the mortality of an animal might have on the landscape. Our framework provides ecologists and conservation practitioners with a dynamic metric for quantifying the ecosystem-wide impacts of roads, and/or the benefits of management strategies. Using movement data from giant anteaters occupying roadside habitats in Brazil, we have demonstrated how this framework can be used in practice to understand the ecological impacts of a road, and help inform species management. Notably, this framework builds straightforwardly off of home-range estimation and requires no specialized data collection protocols, allowing researchers to easily quantify the potential ecological impacts of roads for a variety of taxa. 

We also note that the \texttt{R} scripts required to reproduce these analyses and estimate the probabilistic road effect from animal tracking data are openly available at \url{https://github.com/NoonanM/Road_Effect_Zone}. The Giant anteater tracking data are also openly available on MoveBank (MoveBank ID: 1574830796). However, to ensure anonymity during the double-blind peer-review process, we do not list these in the submitted manuscript. We are happy to provide them though should they be required, 

Our manuscript provides a new theoretical framework for studying the effects of roads on ecosystems. We believe that this work is well suited as a Note in American Naturalist as road ecology research is a broad and impactful topic that will appeal to the readership of your journal.


We look forward to your response.

Kind regards,

Prof. Michael Noonan on behalf of all co-authors

\end{document}